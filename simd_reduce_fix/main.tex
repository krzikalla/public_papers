\newcommand\wgTitle{Fix: Make std::simd Reductions simd-generic.}
\newcommand\wgName{Olaf Krzikalla <olaf.krzikalla@dlr.de>}
\newcommand\wgDocumentNumber{TBD}
\newcommand\wgGroup{LWG}
\newcommand\wgTarget{\CC{}26}
\newcommand\wgAcknowledgements{Thanks to Matthias Kretz.}

\usepackage{mymacros}
\usepackage{wg21}
\setcounter{tocdepth}{2} % show sections and subsections in TOC
\hypersetup{bookmarksdepth=5}
\usepackage{changelog}
\usepackage{underscore}
\usepackage{multirow}

\addbibresource{extra.bib}

\newcommand\simd[1][]{\type{basic\_simd#1}\xspace}
\newcommand\simdT{\type{basic\_simd<T>}\xspace}
\newcommand\valuetype{\type{value\_type}\xspace}
\newcommand\referencetype{\type{reference}\xspace}
\newcommand\simdcast{\code{simd\_cast}\xspace}
\newcommand\mask[1][]{\type{basic\_simd\_mask#1}\xspace}
\newcommand\maskT{\type{basic\_simd\_mask<T>}\xspace}
\newcommand\fixedsizeN{\type{simd\_abi::fixed\_size<N>}\xspace}
\newcommand\fixedsizescoped{\type{simd\_abi::fixed\_size}\xspace}
\newcommand\fixedsize{\type{fixed\_size}\xspace}
\newcommand\wglink[1]{\href{https://wg21.link/#1}{#1}}
\DeclareRobustCommand\simdabi{\code{simd\_abi\MayBreak::\MayBreak}}

\newcommand\nativeabi{\UNSP{native-abi}}
\newcommand\deducet{\UNSP{deduce-abi-t}}
\newcommand\simdsizev{\UNSP{simd-size-v}}
\newcommand\simdsizetype{\UNSP{simd-size-type}}
\newcommand\simdselect{\UNSP{simd-select-impl}}
\newcommand\maskelementsize{\UNSP{mask-element-size}}
\newcommand\integerfrom{\UNSP{integer-from}}
\newcommand\constexprwrapperlike{\UNSP{constexpr-wrapper-like}}
\newcommand\convertflag{\UNSP{convert-flag}}
\newcommand\alignedflag{\UNSP{aligned-flag}}
\newcommand\overalignedflag{\UNSP{overaligned-flag}}
\newcommand\reductionoperation{\UNSP{reduction-binary-operation}}
\newcommand\simdfloatingpoint{\UNSP{simd-floating-point}}
\newcommand\deducedsimd{\UNSP{deduced-simd-t}}
\newcommand\makecompatiblesimdt{\UNSP{make-compatible-simd-t}}
\newcommand\mathfloatingpoint{\UNSP{math-floating-point}}
\newcommand\mathcommonsimd{\UNSP{math-common-simd-t}}

\renewcommand{\lst}[1]{Listing~\ref{#1}}
\renewcommand{\sect}[1]{Section~\ref{#1}}
\renewcommand{\ttref}[1]{Tony~Table~\ref{#1}}
\renewcommand{\tabref}[1]{Table~\ref{#1}}

\begin{document}
\selectlanguage{american}
\begin{wgTitlepage}
  One design goal of std::simd is to enable the writing of simd generic code.
  This is why all arithmetic operators and functions have corresponding overloads.
  However, arithmetic reductions have been forgotten.
\end{wgTitlepage}

\pagestyle{scrheadings}

%\RequirePackage{pifont}

\newcounter{Changelog@Revision}

\def\changelog@base@docnum#1#2#3#4#5R#6{#2#3#4#5}
\def\changelog@p@docnum{P\expandafter\changelog@base@docnum\wgDocumentNumber}
\def\changelog@revision@docnum{\changelog@p@docnum R\arabic{Changelog@Revision}}

\newenvironment{revision}{
  \subsection{Changes from revision \arabic{Changelog@Revision}}
  Previous revision: \href{https://wg21.link/\changelog@revision@docnum}{\changelog@revision@docnum}
  \begin{itemize}
    }{
  \end{itemize}
  \addtocounter{Changelog@Revision}{1}
}

\newcommand\chck{\item[\color{black}\ensuremath{\checkmark}]}
\newcommand\todo{\item[\color{black}\ding{46}] \color{gray}}
\newcommand\itemheader[1]{\item[] \hfill \textcolor{gray}{\textsc{#1}}}


\section{Straw Polls}


\section{Introduction}
\cite{P1928R15} introduced \code{std::simd} and related types and functions.
It enables the programmer to write \emph{simd generic} code, i.e. function templates instantiable
with scalar types as well as vector types.
\medskip\begin{lstlisting}[style=Vc]
template<class T>
auto f(const T& x, const T& y, const T& z)
{
  return x + std::sqrt(y) * std::pow(z);
}
\end{lstlisting}
This function template can be instantiated with scalar floating types, with complex types, or with their
vectorized counterparts (e.g. \code{std::simd<double>} or \code{std::simd<std::complex<double>>}).

Simd generic code is also possible for boolean reductions.
\medskip\begin{lstlisting}[style=Vc]
template<class T>
bool all_lt(const T& x, const T& y)
{
  return std::datapar::all_of(x < y);
}
\end{lstlisting}
Such code is possible because~\cite{P1928R15} explicitly includes overloads for \code{all_of(bool)} aso.

On the other hand, we believe, that~\cite{P1928R15} just forgot to provide scalar overloads for arithmetic reductions.
\medskip\begin{lstlisting}[style=Vc]
template<class T>
auto calc_contribution(const T& x, const T& y)
{
  return std::datapar::reduce(x * y);
}
\end{lstlisting}
With just~\cite{P1928R15} this code is not simd generic yet.
It cannot be called with scalar types, as there is no scalar overload for \code{std::datapar::reduce} yet.
That makes this part of the simd interface incoherent with the rest that does work.

\section{Proposal}

We propose an introduction of scalar overloads for all arithmetic reduce functions introduced in~\cite{P1928R15}.
This applies to the functions in 29.10.7.5: \code{reduce}, \code{reduce_min}, and \code{reduce_max}.
The semantic of the functions is mostly trivial: they just return the passed argument.
The masked functions take a scalar \code{bool} as mask argument.
If the value of that argument is \code{false}, then the functions behave like their vectorized counterparts
if \code{none_of(mask) == true} applies to them.

\section{Discussion}

When the function was still called \code{std::reduce}
there was some doubt whether such an overload of the name could be too much.
But now that reduce moved into the subnamespace \code{std::datapar}
there is no apparent reason to avoid a scalar overload of reduce.

\section{Wording}\label{sec:wording}

The following section presents the wording to be added to the paragraphs introduced in~\cite{P1928R15}.

At the end of §29.10.7.5, add the following entries:
\begin{wgText}
  \lstset{%
    columns=fullflexible,
    deletedelim=**[is]{|-}{-|},
    moredelim=[is][\color{white}\fontsize{0.1pt}{0.1pt}\selectfont{}]{|-}{-|}
  }
  \def\rSec#1[#2]#3{%
  \ifcase#1\wgSubsection[subsection]{#3}{#2}
  \or\wgSubsubsection[subsubsection]{#3}{#2}
  \or\wgSubsubsubsection[paragraph]{#3}{#2}
  \or\error
\fi}

\renewcommand\foralli[1][]{for all $i$ in the range of \range{0}{#1size()}}
\newcommand\Foralli[1][]{For all $i$ in the range of \range{0}{#1size()}}
\renewcommand\forallmaskedi{for all selected indices $i$ of \tcode{mask}}

\newcommand\op{\textrm{\textit{op}}}

\newcommand\ConstraintUnaryOperatorWellFormed[2][const ]{%
  \constraints \tcode{requires (#1value_type a) \{ #2; \}} is \tcode{true}.
}

\newcommand\ConstraintOperatorTWellFormed{%
  \constraints \tcode{requires (value_type a, value_type b) \{ a \op{} b; \}} is \tcode{true}.
}

\newcommand\ConversionsToIntrinsics[1]{
\recommended
Implementations should support explicit conversions between specializations of \tcode{#1} and
appropriate implementation-defined types.
\begin{note}
  The appropriate vector types which are available in the implementation.
\end{note}

%\begin{example}
  %Consider an implementation that supports the type \tcode{__vec4f} and the function \tcode{__vec4f
  %_vec4f_addsub(__vec4f, __vec4f)} for the architecture of the execution environment.
  %A user may require the use of \tcode{_vec4f_addsub} for maximum performance and thus writes:
  %\begin{codeblock}
    %using V = basic_simd<float, simd_abi::__simd128>;
    %V addsub(V a, V b) {
      %return static_cast<V>(_vec4f_addsub(static_cast<__vec4f>(a), static_cast<__vec4f>(b)));
    %}
  %\end{codeblock}
%\end{example}
}


\rSec1[simd.syn]{Header \texorpdfstring{\tcode{<simd>}}{<simd>} synopsis}

%\indexhdr{simd}
\begin{codeblock}
namespace std {

  // \ref{simd.reductions}, \tcode{basic_simd} reductions
  template<class T, class BinaryOperation = plus<>>
    constexpr T reduce(const T&, BinaryOperation = {});
  template<class T, class BinaryOperation = plus<>>
    constexpr T reduce(
      const T& x, same_as<bool> mask,
      BinaryOperation binary_op = {}, type_identity_t<T> identity_element = @\seebelow@);

  template<class T>
    constexpr T reduce_min(const T&) noexcept;
  template<class T>
    constexpr T reduce_min(const T&,
                           same_as<bool>) noexcept;
  template<class T>
    constexpr T reduce_max(const T&) noexcept;
  template<class T>
    constexpr T reduce_max(const T&,
                           same_as<bool>) noexcept;

}
\end{codeblock}

\rSec2[simd.reductions]{\tcode{basic_simd} reductions}

\begin{itemdecl}
template<class T, class BinaryOperation = plus<>>
  constexpr T reduce(const T& x, BinaryOperation binary_op = {});
\end{itemdecl}

\begin{itemdescr}
  \pnum\constraints
  \begin{itemize}
    \item \tcode{T} is vectorizable.

    \item \tcode{BinaryOperation} models \tcode{\reductionoperation<T>}.
  \end{itemize}
  \pnum\expects
  \tcode{binary_op} does not modify \tcode{x}.

  \pnum\returns \tcode{x}

  \pnum\throws
  Any exception thrown from \tcode{binary_op}.
\end{itemdescr}

\begin{itemdecl}
template<class T, class BinaryOperation = plus<>>
  constexpr T reduce(
    const T& x, same_as<bool> mask,
    BinaryOperation binary_op = {}, type_identity_t<T> identity_element = @\seebelow@);
\end{itemdecl}

\begin{itemdescr}
  \pnum\constraints
  \begin{itemize}
    \item \tcode{T} is vectorizable.

    \item \tcode{BinaryOperation} models \tcode{\reductionoperation<T>}.

    \item An argument for \tcode{identity_element} is provided for the invocation, unless
      \tcode{BinaryOperation} is one of \code{plus<>}, \code{multiplies<>}, \code{bit_and<>},
      \code{bit_or<>}, or \code{bit_xor<>}.
  \end{itemize}

  \pnum\expects
  \begin{itemize}
    \item \tcode{binary_op} does not modify \tcode{x}.

    \item For all finite values \tcode{y} representable by \tcode{T}, the results of
      \tcode{y == binary_op(simd<T, 1>(identity_element), simd<T, 1>(y))[0]} and
      \tcode{y == binary_op(simd<T, 1>(y), simd<T, 1>(identity_element))[0]} are \tcode{true}.
  \end{itemize}

  \pnum\returns
  If \tcode{mask} is \tcode{false}, returns \tcode{identity_element}.
  Otherwise, returns \tcode{x}.

  \pnum\throws
  Any exception thrown from \tcode{binary_op}.

  \pnum\remarks
  The default argument for \code{identity_element} is equal to
  \begin{itemize}
    \item \tcode{T()} if \code{BinaryOperation} is \code{plus<>},
    \item \tcode{T(1)} if \code{BinaryOperation} is \code{multiplies<>},
    \item \tcode{T(\~{}T())} if \code{BinaryOperation} is \code{bit_and<>},
    \item \tcode{T()} if \code{BinaryOperation} is \code{bit_or<>}, or
    \item \tcode{T()} if \code{BinaryOperation} is \code{bit_xor<>}.
  \end{itemize}
\end{itemdescr}

\begin{itemdecl}
template<class T> constexpr T reduce_min(const T& x) noexcept;
\end{itemdecl}

\begin{itemdescr}
  \pnum\constraints
  \begin{itemize}
    \item \tcode{T} is vectorizable.

    \item \tcode{T} models \tcode{totally_ordered}.
  \end{itemize}

  \pnum\returns \tcode{x}.
\end{itemdescr}

\begin{itemdecl}
template<class T>
  constexpr T reduce_min(
    const T& x, same_as<bool> mask) noexcept;
\end{itemdecl}

\begin{itemdescr}
  \pnum\constraints
  \begin{itemize}
    \item \tcode{T} is vectorizable.

    \item \tcode{T} models \tcode{totally_ordered}.
  \end{itemize}

  \pnum\returns
  If \tcode{mask} is \tcode{false}, returns \tcode{numeric_limits<T>::max()}.
  Otherwise, returns \tcode{x}.
\end{itemdescr}

\begin{itemdecl}
template<class T> constexpr T reduce_max(const T& x) noexcept;
\end{itemdecl}

\begin{itemdescr}
  \pnum\constraints
  \begin{itemize}
    \item \tcode{T} is vectorizable.

    \item \tcode{T} models \tcode{totally_ordered}.
  \end{itemize}

  \pnum\returns \tcode{x}.
\end{itemdescr}

\begin{itemdecl}
template<class T>
  constexpr T reduce_max(
    const T& x, same_as<bool> mask) noexcept;
\end{itemdecl}

\begin{itemdescr}
  \pnum\constraints
  \begin{itemize}
    \item \tcode{T} is vectorizable.

    \item \tcode{T} models \tcode{totally_ordered}.
  \end{itemize}

  \pnum\returns
  If \tcode{mask} is \tcode{false}, returns \tcode{numeric_limits<T>::lowest()}.
  Otherwise, returns \tcode{x}.
\end{itemdescr}

% vim: tw=100 cc=99

\end{wgText}

\end{document}
% vim: sw=2 sts=2 ai et tw=0
